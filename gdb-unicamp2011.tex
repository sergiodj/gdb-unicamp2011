\documentclass[xcolor=pdftex,dvipsnames,table,t]{beamer}

\usepackage{ae,aecompl}
\usepackage[brazil]{babel}
\usepackage[T1]{fontenc}
%\usepackage{listings}
%\lstset{language=sh,showstringspaces=true,showtabs=true}

\usepackage{url}
\usepackage{hyperref}
%\usepackage{beamerthemesplit}
\usepackage{xcolor}
%\usepackage{multirow}
\usepackage[utf8]{inputenc}

%\usetheme{Boadilla}
%\usetheme{Rochester}
%\usetheme{CambridgeUS}

%\usetheme{Copenhagen}
%\usecolortheme{rose}

\usetheme{Darmstadt}
\usecolortheme{dolphin}

%\usetheme{Warsaw}
%\usecolortheme{whale}

%\usetheme[secheader]{Madrid}

\newcommand{\parametro}[1]{$<$#1$>$}

\title{Aprendendo a depurar programas com o GDB}
\author{Edjunior Machado \and
	Sérgio Durigan Júnior}
\date[Unicamp]{28 de Novembro de 2011}

\begin{document}

\frame{
	\titlepage
	\begin{center}
		\includegraphics[width=0.5\textwidth]{images/gdb-logo.png}
	\end{center}
}

\section{Introdução}
\begin{frame}
	\frametitle{Compilando com informações para depuração}
	  \begin{itemize}
	    \item \texttt{\$ gcc -g3 fonte.c -o programa}
	    \item \texttt{\$ CFLAGS="-g3" ./configure; make}
	  \end{itemize}
\end{frame}

\begin{frame}
	\frametitle{Rodando GDB}
	  \begin{itemize}
	    \item \texttt{\$ gdb ./programa}
	    \item \texttt{\$ gdb --args ./programa <arg1> <arg2>}
	    \item \texttt{\$ gdb \\
		(gdb) run <arg2> <arg2> ...}
	  \end{itemize}
\end{frame}

\begin{frame}
	\frametitle{Parando e Continuando} %FIXME
	  \begin{itemize}
	    \item Breakpoints
	      \begin{itemize}
		\item Breakpoints temporários
	      \end{itemize}
	    \item Watchpoints
	      \begin{itemize}
		\item read, access, write
		\item Watchpoints condicionais
	      \end{itemize}
	    \item Catchpoints
	      \begin{itemize}
		\item \texttt{catch syscall}
	      \end{itemize}
	  \end{itemize}
\end{frame}

\begin{frame}
	\frametitle{Parando e Continuando}
	  \begin{itemize}
	    \item \texttt{continue}
	    \item \texttt{next}, \texttt{nexti}
	    \item \texttt{step}, \texttt{stepi}
	    \item \texttt{finish}
	  \end{itemize}
\end{frame}

\begin{frame}
	\frametitle{Examinando o código}
	  \begin{itemize}
	    \item \texttt{list}
	    \item \texttt{disassemble}
	    \item Especifique onde está o código fonte através do comando \texttt{dir <local>}
	  \end{itemize}
\end{frame}

\begin{frame}
	\frametitle{TUI (Text User Interface)}
	  \begin{itemize}
	    \item Embora não muita difundida, é uma interface bastante útil
	    \item Pressione CTRL+x a (ou CTRL+x 1 ou CRTL+x 2)
	    \item Pode ter alguns problemas caso o programa sendo depurado utilize a saida padrão
	  \end{itemize}
\end{frame}

\begin{frame}
	\frametitle{Examinando dados}
	  \begin{itemize}
	    \item \texttt{print <var>}
	    \begin{itemize}
	      \item \texttt{p *array@<elementos>}
	    \end{itemize}
	    \item \texttt{x <var>}
	    \item \texttt{whatis <var>}
	    \item \texttt{display <var>}
	    \item Configure a exibição de \texttt{structs} através de \\
		  \texttt{set print pretty on}
	  \end{itemize}
\end{frame}

\begin{frame}
	\frametitle{Examinando dados}
	  \begin{itemize}
	    \item \texttt{bt}
	    \item \texttt{frame}
	    \item \texttt{up}, \texttt{down}
	  \end{itemize}
\end{frame}

\begin{frame}
	\frametitle{Alterando o programa depurado}
	  \begin{itemize}
	    \item Alterando os dados
	      \begin{itemize}
		\item \texttt{set var variavel = <valor>}
		\item \texttt{set \{int\}0x9876 = 666}
	      \end{itemize}
	    \item Alterando o fluxo
	      \begin{itemize}
		\item \texttt{jump}
		\item \texttt{return}
	      \end{itemize}
	  \end{itemize}
\end{frame}

\begin{frame}
	\frametitle{Corefiles} %TODO maiores explicações?
	  \begin{itemize}
	  \item Habilite a criação de \texttt{corefiles} com \\
		\texttt{\$ ulimit -c unlimited}
	  \item Rodando o GDB \\
		\texttt{\$ gdb programa core}
	  \item De dentro do GDB \\
		\texttt{generate-core-file} \\
		\texttt{core}
	  \end{itemize}
\end{frame}

\begin{frame}
	\frametitle{Outras informações}
	  \begin{itemize}
	    \item \texttt{info}
	    \begin{itemize}
	      \item \texttt{info registers}
	    \end{itemize}
	  \end{itemize}
\end{frame}

\begin{frame}
	\frametitle{Em dúvida sobre algum comando?}
	  \begin{itemize}
	    \item \texttt{help}
	  \end{itemize}
\end{frame}

\begin{frame}
	\frametitle{Outras funcionalidades}
	  \begin{itemize}
	    \item Tracepoints
	    \item \texttt{forks} e \texttt{threads}
	    \item Depuração reversa
	    \item \texttt{gdbserver}
	  \end{itemize}
\end{frame}

\section{Referências}
\begin{frame}
       \frametitle{Referências}
        \begin{center}
        \begin{itemize}
		\item \textbf{Debugging with GDB} \\
		http://sourceware.org/gdb/current/onlinedocs/gdb/
	\end{itemize}
        \end{center}
\end{frame}

\section{}
\begin{frame}
	\begin{center}
	\LARGE
	\alert{Obrigado!}

	Dúvidas?


	\vspace{2\baselineskip}

	\small
	Edjunior Machado \\
	{\tt edjunior@gmail.com} \\
	\vspace{1\baselineskip}
	Sérgio Durigan Júnior \\
	{\tt sergiosdj@gmail.com}
	\end{center}
\end{frame}

\end{document}

