\documentclass[xcolor=pdftex,dvipsnames,table,t]{beamer}

\usepackage{ae,aecompl}
\usepackage[brazil]{babel}
\usepackage[T1]{fontenc}
%\usepackage{listings}
%\lstset{language=sh,showstringspaces=true,showtabs=true}

\usepackage{url}
\usepackage{hyperref}
%\usepackage{beamerthemesplit}
\usepackage{xcolor}
%\usepackage{multirow}
\usepackage[utf8]{inputenc}

%\usetheme{Boadilla}
%\usetheme{Rochester}
%\usetheme{CambridgeUS}

%\usetheme{Copenhagen}
%\usecolortheme{rose}

\usetheme{Darmstadt}
\usecolortheme{dolphin}

%\usetheme{Warsaw}
%\usecolortheme{whale}

%\usetheme[secheader]{Madrid}

\newcommand{\parametro}[1]{$<$#1$>$}

\title{Aprendendo a depurar programas com o GDB}
\author{Edjunior Machado \and
	Sérgio Durigan Júnior}
\date[Unicamp]{\today}

\begin{document}

\frame{
	\titlepage
	\begin{center}
		\includegraphics[width=0.5\textwidth]{images/gdb-logo.png}
	\end{center}
}

\begin{frame}
	\frametitle{Licença}
	\begin{itemize}
		\item{Licença: \textbf{Attribution-ShareAlike 3.0 Unported (CC BY-SA 3.0)}}
		\item{\url{http://creativecommons.org/licenses/by-sa/3.0/}}
	\end{itemize}
\end{frame}

\begin{frame}[fragile]
	\frametitle{Antes de mais nada...}
	\begin{itemize}
	  \item{O \verb|TAB| é seu amigo}
	  \item{\verb|help| e \verb|apropos| também}
	  \item{A tecla \verb|enter| repete o comando anterior}
	\end{itemize}
\end{frame}

% Para usar \verb||, o frame tem que ser marcado como fragile.
\begin{frame}[fragile]
	\frametitle{Compilando o Programa com Informações para Depuração}
	  \begin{itemize}
	    \item{\verb|gcc -g3 -O0 fonte.c -o programa|}
	    \item{\verb|CFLAGS="-g3 -O0" ./configure; make|}
	      \begin{itemize}
		\item{O segundo parâmetro da variável \verb|CFLAGS| é \textbf{\textit{menos ó zero}}}
	      \end{itemize}
	  \end{itemize}
\end{frame}

\begin{frame}[fragile]
	\frametitle{Executando o GDB}
	  \begin{itemize}
	    \item{\verb|gdb ./programa|}
	    \item{\verb|gdb --args ./programa <arg1> <arg2>|}
	    \item{\verb|gdb| \\
		\verb|(gdb) file programa| \\
		\verb|(gdb) run <arg2> <arg2> ...|\\
		ou\\
		\verb|(gdb) start <arg2> <arg2> ...|}
	  \end{itemize}
\end{frame}

\begin{frame}[fragile]
	\frametitle{Examinando Dados}
	  \begin{itemize}
	    \item{\verb|print <variável>|}
	    \begin{itemize}
	      \item{\verb|p *array@<elementos>|}
	    \end{itemize}
	    \item{\verb|x <endereço>|}
	    \item{\verb|whatis <variável>|}
	    \item{\verb|display <variável>| e \verb|info display|}
	    \item{Configure a exibição de \verb|structs| através de \\
		  \verb|set print pretty on|}
	  \end{itemize}
\end{frame}

\begin{frame}[fragile]
	\frametitle{TUI (Text User Interface)}
	  \begin{itemize}
	    \item{Embora não muita difundida, é uma interface bastante útil}
	    \item{Pressione \verb|CTRL+x a| (ou \verb|CTRL+x 1| ou \verb|CTRL+x 2|)}
	    \item{Pode ter alguns problemas caso o programa sendo depurado utilize a saida padrão ou a janela do terminal seja redimensionada}
	    \item{\textit{Workaround}: sair e entrar novamente na interface}
	  \end{itemize}
\end{frame}

\begin{frame}[fragile]
	\frametitle{Parando e Continuando a Execução}
	  \begin{itemize}
	    \item{Breakpoints}
	      \begin{itemize}
		\item{\verb|break|}
		  \begin{itemize}
		    \item{\verb|break 7| (parar na linha 7)}
		    \item{\verb|break foo| (parar na função \verb|foo|)}
		    \item{\verb|break fonte.c:bar| (parar na função \verb|bar|, no arquivo \verb|fonte.c|)}
		    \item{\verb|break fonte.c:7| (parar na linha 7 do arquivo \verb|fonte.c|)}
		  \end{itemize}
		\item{\verb|tbreak| (breakpoint temporário); mesmos argumentos que \verb|break|}
	      \end{itemize}
	    \item{Watchpoints}
	      \begin{itemize}
		\item{\verb|watch| (escrita), \verb|rwatch| (leitura) e \verb|awatch| (acesso)}
		\item{Watchpoints condicionais}
	      \end{itemize}
	    \item{Catchpoints}
	      \begin{itemize}
		\item{\verb|catch syscall|}
		\item{\verb|catch fork|, com \verb|follow-fork-mode|}
	      \end{itemize}
	  \end{itemize}
\end{frame}

\begin{frame}[fragile]
	\frametitle{Parando e Continuando a Execução$^2$}
	  \begin{itemize}
	    \item{\verb|continue|}
	    \item{\verb|next|, \verb|nexti|}
	    \item{\verb|step|, \verb|stepi|}
	    \item{\verb|finish|}
	    \item{Loop infinito? \verb|CTRL+c|}
	  \end{itemize}
\end{frame}

\begin{frame}[fragile]
	\frametitle{Examinando o Código}
	  \begin{itemize}
	    \item{\verb|list|}
	    \item{\verb|disassemble|}
	  \end{itemize}
\end{frame}

\begin{frame}[fragile]
	\frametitle{Examinando Dados$^2$}
	  \begin{itemize}
	    \item{\verb|bt|}
	    \item{\verb|frame|}
	    \item{\verb|up|, \verb|down|}
	  \end{itemize}
\end{frame}

\begin{frame}[fragile]
	\frametitle{Alterando o Programa Depurado}
	  \begin{itemize}
	    \item{Alterando os Dados}
	      \begin{itemize}
		\item{\verb|set var <variável> = <valor>|}
		\item{\verb|set {int}<endereço> = <valor>|}
	      \end{itemize}
	    \item{Alterando o Fluxo}
	      \begin{itemize}
		\item{\verb|jump|}
		\item{\verb|return|}
	      \end{itemize}
	  \end{itemize}
\end{frame}

\begin{frame}[fragile]
	\frametitle{Corefiles}
	  \begin{itemize}
	  \item{Representação do estado do um programa em determinado momento}
	  \item{Pode ser gerado manualmente (GDB) ou automaticamente em caso de falha do programa}
	  \item{Habilite a criação de \verb|corefiles| com \\
		\verb|ulimit -c unlimited|}
	  \item{Executando o GDB \\
		\verb|gdb programa core.PID|}
	  \item{Manualmente, de dentro do GDB \\
		\verb|generate-core-file| \\
		\verb|core|}
	  \end{itemize}
\end{frame}

\begin{frame}[fragile]
	\frametitle{Outras Informações}
	  \begin{itemize}
	    \item{\verb|info|}
	    \begin{itemize}
	      \item{\verb|info breakpoints|}
	      \item{\verb|info watchpoints|}
	      \item{\verb|info locals|}
	      \item{\verb|info registers|}
	      \item{\verb|...|}
	    \end{itemize}
	  \end{itemize}
\end{frame}

\begin{frame}[fragile]
	\frametitle{Depuração Reversa(!!)}
	  \begin{itemize}
	    \item{Suportada desde 2008 (mas pouca gente sabe)}
	    \item{\verb|(gdb) start| \\
		  \verb|(gdb) target record| \\
		  \verb|(gdb) next| \\
		  \verb|(gdb) ...| \\
		  \verb|(gdb) reverse-next|}
	    \item{Altere a direção de execução com \verb|set exec-direction [forward,reverse]|}
	  \end{itemize}
\end{frame}

\begin{frame}[fragile]
	\frametitle{Em Dúvida Sobre Algum Comando?}
	  \begin{itemize}
	    \item{\verb|help|}
	    \item{\verb|apropos|}
	  \end{itemize}
\end{frame}

\begin{frame}[fragile]
	\frametitle{Outras Funcionalidades}
	  \begin{itemize}
	    \item{\verb|Tracepoints|}
	    \item{\verb|forks| e \verb|threads|}
	    \item{\verb|gdbserver|}
	  \end{itemize}
\end{frame}

\begin{frame}
       \frametitle{Referências}
        \begin{center}
        \begin{itemize}
		\item \textbf{GDB: The GNU Project Debugger} \\
		http://sourceware.org/gdb/
		\item \textbf{Debugging with GDB} \\
		http://sourceware.org/gdb/current/onlinedocs/gdb/
		\item \textbf{GDB Wiki} \\
		http://sourceware.org/gdb/wiki/
		\item canal \texttt{\#gdb} na Freenode
	\end{itemize}
        \end{center}
\end{frame}

\section{}
\begin{frame}
	\begin{center}
	\LARGE
	\alert{Obrigado!}

	Dúvidas?


	\vspace{2\baselineskip}

	\small
	Edjunior Machado \\
	{\tt edjunior@gmail.com} \\
	\vspace{1\baselineskip}
	Sérgio Durigan Júnior \\
	{\tt sergiodj@redhat.com}
	\end{center}
\end{frame}

\end{document}

